% !TEX root = SystemTemplate.tex
\chapter{User Stories, Backlog and Requirements}
\section{Overview}
This section contains several user stories, a backlog, and a list of requirements, of
both the project and the user, and the user's equiptment for using the project. This
chapter will contain 
details about each of the requirements and how the requirements are or will be 
satisfied in the design and implementation of the system.

The user stories are provided by the stakeholders.

\subsection{Scope}

This document will contain stakeholder information, initial user stories, requirements, and 
proof of concept results.



\subsection{Purpose of the System}
The purpose of the product is to allow the user to run pre-written and product generated auto tests on a directory or 
program of their choosing, and to provide a log to the user of which tests passed and which tests failed for each program 
that is found within the given directory. A PASS, FAIL, and or percentage is written to a log placed within a found 
program's directory. 

In addition, the purpose of the product is to allow the user to generate a given number of random 
tests with user given parameters.

\section{ Stakeholder Information}


\subsection{Customer or End User (Product Owner)}
Whitespace Cowboys:
The Product Owner is Ryan Brown. The Product Owner in this case is responsible for 
getting project specifications from the Customer and keeping the team members up to date 
with the Customer's user stories. The Product Owner is also responsible for managing and 
prioritizing the product backlog.

Obfuscators:
Daniel Nix is the Product Owner of this project.  He will clarify and define the End Users' needs and requirements 
for this product, as well as establish and prioritize the product backlog.


Kernel\_Panic:
The Product Owner is Benjamin Sherman. He will relay the customer specifications and requirements.
 
\subsection{Management or Instructor (Scrum Master)}
Whitespace Cowboys:
The Scrum Master is Kelsey Bellew, and will drive the Sprint Meetings, keep a log of
meetings and schedules. She will also deal with any and all unforseen issues causing 
dilemmas to the completion of the project.

Obfuscators:
Lisa Woody is the Scrum Master for the project.  She is responsible for scheduling the project meetings, as well as 
determining and assigning the tasks necessary to deliver the required product.

Kernel\_Panic: The Scrum Master is James Tillma. He is in charge of driving the sprint meetings, setting deadlines, and handling the unforeseen obstacles that may be encountered.

\subsection{Developers --Testers}
Whitespace Cowboys:
The Technical Lead for our team is Ryan Feather.  He coordinates the merging of user stories and ensures
 program stability throughout development.

Obfuscators:
Joseph Lillo is the Technical Lead and for the project.  He will be responsible for the high level design and final
testing of the program.

Kernel\_Panic:
Anthony Morast is the teams Technical Lead. He will be responsible for the maintaining product stability throughout the development process.


\section{Business Need}
This software must simplify and automate the grading process.  The product will meet that need and enable 
the end user to not only see the immediate results of a test, but also to maintain a dated record of each test
and its detailed output.

\section{Requirements and Design Constraints}

\subsection{System  Requirements}
\begin{description}
\item [$\bullet$] The program must build and run in a Linux environment.
\item [$\bullet$] The source code for programs to be "tested" will be in C++.
\item [$\bullet$] Source code will be in the "root" directory and its subdirectories.
\item [$\bullet$] A bash shell will be used to run the program
\end{description}

\subsection{Network Requirements}
There are no network requirements. This project does not use the internet unless the program it is 
testing uses the internet.


\subsection{Development Environment Requirements}
\begin{description}
\item [$\bullet$] The application must be written in C++
\item [$\bullet$] All work must be done in Linux
\item [$\bullet$] System calls (gcc, etc.) may be used in the application.
\item [$\bullet$] The test case input will be stored in a .tst file. The accompanying
 desired output \\ will be stored in a .ans file.
\item [$\bullet$] Testing output should be in one log file which contains: \\
\hspace{4ex} Test output results (i.e., 52 different tests will produce 52 lines in the .log file) \\
\hspace{4ex} Number passed, Number failed, Percentage of success
\item[$\bullet$] Code coverage percentage  will be each student detailed log file.
\item[$\bullet$] Code profile information from a student program will be located in the same directory as a students source code.
\end{description}

\subsection{Project  Management Methodology}
There is only one customer for this application. This customer may place constraints
on meeting times and frequency of required progress reports. Aside from customer
requests meeting times and reports will be managed by the scrum master.
For the first iteration, we need to compile and test only one program. 

\begin{itemize}
\item Trello, a free web-based project management application, will be used to keep
         track of the backlogs and sprint status.
\item All parties have access to the Sprint and Product Backlogs, via Trello.
\item This particular project will be encompassed by only one Sprint.
\item The Sprint Cycle of this project is three weeks.
\item There are no restrictions on source control.
\end{itemize}

\section{User Stories}


\subsection{User Story \#1}
As a user of the program, I would like to be able to specify a program to grade that will test the program and create a record of easy to understand output.

\subsubsection{User Story \#1 Breakdown}
This application will be targeted towards instructors needing to test submitted student programs against applicable test
cases.  The application will be run from the command line, using the name of the program to be tested as an inital parameter.
For each existing test case, the program will be run using that test case's .tst file as input.  The output will be recorded and 
compared to the accompanying answer file for that test case.  A summary of the results must accompany the recorded
output in the log file created each time the application is run.

\subsection{User Story \#2} 

As a user of the program, I want to be able to test the program against test cases located in the directory tree of that program. 


\subsubsection{User Story \#2 Breakdown}
Each program to be tested will be placed into a directory that forms the root of the directory tree related to that program.
The user will have the ability to add and remove test cases  (called case\#.tst) and their accompanying desired output file
for comparison (called case\#.ans).  The application must find all of the applicable test cases and accompanying answer 
files, and test the specified program against all the test cases found.

\subsection{User Story \#3} 
As a user of the program, I want to be able to fix the problems in the program I am testing and rerun the test without losing
the previously created log file.

\subsubsection{User Story \#3 Breakdown}
A new log file containing the tested program's outputs and summary must be created each time the application is run.
This will be an important feature, enabling the user to alter the program and visualize the effects of the alterations on
the program's output and testing summary.  Each log file will be date-stamped to achieve this result.

\subsection{User Story \#4}
User wants to be able to run multiple tests on multiple programs.

\subsection{User Story \#4 Breakdown}
The Auto Tester must be able to take a directory, and do a directory crawl while finding all .tst and .ans files contained 
within the given directory, and then do a second directory crawl to find all programs and run all tests agains all found 
programs. 

\subsection{User Story \#5}
User wants to be able to run the Auto Tester without giving it specific program(s) to test.

\subsection{User Story \#5 Breakdown}
The Auto Tester must be able to find a program or programs just given a directory, and must be able to differenciate 
between test files, answer files, student programs, and the golden cpp.

\subsection{User Story \#6}
User wants to be able to have the option to have the tester auto generate tests.

\subsection{User Story \#6 Breakdown}
The Auto Tester will prompt the user if they want to auto generate a test; they will be prompted to enter different 
parameters which will be used to generate a random .tst, the contents of which will be run against the golden cpp to 
generate a .ans file.

\subsection{User Story \#7}
User wants to be able to set certian tests as 'must pass' tests; and if these 
tests are not passed, then the student is not given a percentage, but rather a FAIL.

\subsection{User Story \#7 Breakdown}
The product must be able to detect if a test is marked as a 'must pass' test, and when the Auto Tester 
runs that test against a student's program and that program happens to fail that test, the Auto Tester 
must be able to recognize the failure of a critical test and cease the running of any further tests. 
It must then not give the program a percentage of passed tests as it might for other student programs, 
or as it did in the previous itteration of this product, and only output 'FAIL'.

\subsection{User Story \#8}
As a user of the product I want to see the percentage of a students code that was exercised (tested) by the test cases in a students log file.

\subsection{User Story \#8 Breakdown}
The product will use GCOV to monitor and retrieve code coverage information. This will not be an option but will be stored in each students log file.

\subsection{User Story \#9}
As a user I want to be able to allow leniency on how close a students program output and the answer must match.

\subsection{User Story \#9 Breakdown}
The user of the product can choose whether to allow slight formatting errors on a students program output. There can be minor differences between a test case answer and a student program answer.

\subsection{User Story \#10}
As a user I want to be able to fail a student on a test case if their code goes into an infinite loop.

\subsection{User Story \#10 Breakdown}
That this appears to be the unsolvable halting problem is not the case. Instead the user of the product will prompted for an allowable time period that a student program will be given to run one test case. If a students program doesn't complete in the specified time, then it will fail the test case under an assumed infinite loop.

\section{Research or Proof of Concept Results}
This section is reserved for the discussion centered on any research that needed 
to take place before full system design.  The research efforts may have led to 
the need to actually provide a proof of concept for approval by the stakeholders. 
 The proof of concept might even go to the extent of a user interface design or 
mockups.


\section{Supporting Material}
This document might contain references or supporting material which should be documented 
and discussed  either here if approprite or more often in the appendices at the end.  
This material may have been provided by the stakeholders  
or it may be material garnered from research tasks.
