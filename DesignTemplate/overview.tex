% !TEX root = SystemTemplate.tex

\chapter{Overview and concept of operations}
This program is a utility for the testing of student software. It takes a students program 
and checks if it conforms to a suite of test cases. This document is included on the 
submission in order to facilitate use and maintenance of the utility. It covers the 
design, implementation, and usage of the provided software.


\section{Scope}
The scope of this document is meant to be comprehensive, giving users and managers the 
information necessary to use the product.


\section{Purpose}
The purpose of the Auto Tester is to allow batch-grading. Specifically, this program 
will traverse through a given directory, and run input through any found .cpps and 
test if the output is what was expected by the instructor. A summary file will be 
generated for each program found, and for each run of the Auto Tester, 
detailing which tests passed or failed, the percentage each program achieved overall, and/or 
a PASS or FAIL to dictate whether the student passed or failed the program. As well as the percentage of code that was tested by the test cases.

 In addition to this, 
the Auto Tester will allow the user to easily generate random .tst files with the assosiated .ans 
file, given that they provide the Auto Tester with a 'golden cpp'.

\subsection{Compiling}
The product will be compiled using the g++ GNU linux compiler
version 4.8.2. Provided is a Makefile that allows \textit{make} to be typed into a terminal. After this is completed, the product will be ready for use.


\subsection{Identifying Test Cases and Inputting}
Once the student program is compiled, the Auto Tester searches the current directory and all 
sub-directories for files labeled as test cases (a .tst extension). Each test file is then 
used as input for the student program.


\subsection{Checking Output}
The Auto Tester will re-direct the programs output and evaluate it against the supplied test case. 
The results of the comparison (pass/fail) are recorded in a log file for each student, for each 
test case encountered.



\subsection{Major System Component \#1}
Location of existing, applicable test cases for the program needing to be tested.

\subsection{Major System Component \#2}
Run and test of a single program against located test cases

\subsection{Major System Component \#3}
Record and summary of test results

\section{Systems Goals}
The goal of this system is to provide an automated testing application, designed specifically
for professors testing submitted student programs. A user will be able to use the application
to test a desired program against all applicable test cases the application can find in the directory tree 
related to that program.  A time-stamped record will be created for each program found, to summarize the 
output of each test and to provide a general summary of the results as well as the percentage of code covered by the test cases. In addition, a summary log file will be 
generated including the names of the students, and what percent they got on the assignment, or if the simply 
failed the assignment.

The user (professor) will be prompted for \textit{allowable time} which is the amount of time that a student program will be allowed to run before it is considered as in an infinite loop. When a student program is classified as being in an infinite loop it will be marked as failed.

Also, the use (professor) can choose to generate profiling information. When this option is enabled, the product will keep track of the amount of time spent in sub functions of student programs and supply this information in a file called \textit{profile\textless timestamp\textgreater .out} which will be located in the same directory as a student source code.

\section{System Overview and Diagram}
The major system components listed above will, upon completion, combine to create this testing application. 
Upon running the application and providing it with the name of the desired target program, the application
will complete the desired system goals via its major components.
\\ Initially options will be gathered from the user. These options are the \textit{allowable time},\textit{ profiling enabled}, and whether or not they wish to \textit{generate test cases}.
\\ First, existing, applicable test cases will be found for the program needing to be tested. 
\\ Second, the program will be run against each test case input, and the resulting output will be 
compared to each desired test case output.
\\  Last, the program will create a time-stamped record for each program tested, providing a reference of 
the output results and a numerical summary of the overall success rate, the percentage of the code tested by the test cases (\textit{code coverage}), and, if enabled, run time statistics (\textit{profiling}).  
\\ See Figure~\ref{systemdiagram}.
\begin{figure}[H]
\begin{center}
\includegraphics[width=1.0\textwidth]{./SystemDiagram}
\end{center}
\caption{The Testing System Diagram \label{systemdiagram}}
\end{figure}



\section{Technologies Overview}
The development team used the Agile Software Development Method, via the Scrum framework, to develop
this system.  This incremental development method was the required development method for this project.
Since the system was expected was built for a Linux environment, it was written and tested on a Linux Platform 
using generic text editors and the G++ compiler. 

Integrated into the product was GCOV. GCOV was used to obtain the amount of student source code that was exercised by the test cases.

An optional technology used in the product was GPROF which is implemented when the user enables code profiling. 


See Table~\ref{DevelopmentTable}.  
\begin{table}[tbh]
\begin{center}
\begin{tabular}{|r|l|}
  \hline
  Software Development Method & Agile Software Development Method \\
  Planning and Organization & Trello Project Management Application \\
  \hline \hline
  Platform & Linux \\
  Language & C++ \\ 
  Code Coverage & GCOV \\
  Code Profiling & GPROF \\ 
  IDE & none (generic text editors) \\
  Version Control & Git
  \\ \cline{2-2}
  
  \hline
\end{tabular}
\caption{The Development Methods and Technologies Table \label{DevelopmentTable}}
\end{center}
\end{table}

